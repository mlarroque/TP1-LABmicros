\section{Menú Principal}

Al comenzar, el display mostrará la palabra \texttt{MENU}. 
Con el encoder se podrá elegir, y confirmar la opción con un enter del encoder, entre \texttt{ID}, 
ya sea para ingresar manualmente el ID o \texttt{GLOW}, para cambiar la intensidad.

\subsection{Intensidad del display}

Dentro del submenú \texttt{GLOW}, se podrá elegir entre \texttt{25\%}, \texttt{50\%}, \texttt{75\%} y \texttt{100\%}, 
correspondiéndose 25\% con la mínima intensidad, y 100\% con la máxima intensidad del display.

\section{ID}
\subsection{Ingreso Manual con el Encoder}
\label{pinsection}
El código identificatorio de 8 dígitos se puede ingresar manualmente, con el encoder, 
seleccionando entre \texttt{0-9} para los dígitos. Además, el usuario podrá elegir letras,
 \texttt{L} ó \texttt{A}; con \texttt{L} (\texttt{erase \underline{L}ast}) para borrar el último dígito, 
 y \texttt{A} (\texttt{erase \underline{A}ll}) para borrar todo el ID.
 Para confirmar el dígito, hacer click con el encoder.
 Si el ID se ingresó correctamente, aparecerá el mensaje \texttt{valid ID}, seguido de \texttt{enter PIN}.
 De no ser correcto el ID ingresado, aparecerá el mensaje \texttt{invalid ID}, debiéndose ingresar nuevamente.

 \subsection{Ingreso Automático con Lector de Tarjeta Magnética}
 Para el ingreso automático, pasar la tarjeta con la orientación indicada en el lector.
 De ser correcta la lectura, aparecerá el mensaje \texttt{valid ID}, seguido de \texttt{enter PIN}.
 De no ser correcta la lectura, aparecerá el mensaje \texttt{invalid ID}.

 \section{PIN}
 El PIN se ingresa manualmente con el encoder, de la misma forma 
 que se indicó en la sección \ref{pinsection}. Tiene 4 o 5 dígitos.
 Si se ingresó de manera correcta, aparecerá el mensaje \texttt{valid PIN}, 
 y se garantizará el acceso al usuario.
Si se ingresó de manera incorrecta, aparecerá el mensaje \texttt{invalid PIN}.
Tras 3 intentos de ingreso de PIN incorrectos, se bloquea el usuario, 
y deberá esperar un tiempo para poder reingresarlo, incrementándose en cada nuevo intento.

\section{Acceso Garantizado}
\subsection{Usuario no Administrador}
\label{noadminmenu}
Si el usuario no es administrador, podrá elegir entre \texttt{OPEN}, abrir la puerta; y \texttt{PIN},
para cambiar su PIN. Para esta última opción, referirse a la sección \ref{pinsection} para 
el modo de ingreso manual del nuevo PIN.
\subsection{Usuario Administrador}
 \label{adminmenu}
 Si el usuario es administrador, tendrá las opciones \texttt{OPEN}, para abrir la puerta; \texttt{PIN}, 
 para cambiar su PIN, ver sección \ref{noadminmenu} para el ingreso manual del nuevo PIN;
  y \texttt{ADM}, para agregar o eliminar usuarios.\par
  Al elegir esta última opción, se puede elegir entre \texttt{DLT}, para eliminar un usuario, ingresando su ID,
  de la manera especificada en la sección \ref{pinsection} o \texttt{ADD}, para agregar un usuario 
  ingresando su ID y a continuación, su PIN. Si el usuario ya es existente en la base de datos, 
  aparecerá el mensaje \texttt{USER EXISTS}.




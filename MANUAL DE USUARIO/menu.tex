\section{Menú Principal}

Al comenzar, el display mostrará la palabra $MENU$. 
Con el encoder se podrá elegir, y confirmar la opción con un enter del encoder, entre $ID$, 
ya sea para ingresar manualmente el ID o $GLOW$, para cambiar la intensidad.

\subsection{Intensidad del display}

Dentro del submenú $GLOW$, se podrá elegir entre 25\%, 50\%, 75\% y 100\%, 
correspondiéndose 25\% con la mínima intensidad, y 100\% con la máxima intensidad del display.

\section{ID}
\subsection{Ingreso Manual con el Encoder}
\label{pinsection}
El código identificatorio de 8 dígitos se puede ingresar manualmente, con el encoder, 
seleccionando entre $0-9$ para los dígitos. Además, el usuario podrá elegir letras, $L$ ó $A$;
con $L$ (erase \underline{L}ast) para borrar el último dígito, y $A$ (erase \underline{A}ll)
 para borrar todo el ID.
 Para confirmar el dígito, hacer click con el encoder.
 Si el ID se ingresó correctamente, aparecerá el mensaje "valid ID", seguido de "enter PIN".
 De no ser correcto el ID ingresado, aparecerá el mensaje "invalid ID".

 \subsection{Ingreso Automático con Lector de Tarjeta Magnética}
 Para el ingreso automático, pasar la tarjeta con la orientación indicada en el lector.
 De ser correcta la lectura, aparecerá el mensaje "valid ID", seguido de "enter PIN".
 De no ser correcta la lectura, aparecerá el mensaje "invalid ID".

 \section{PIN}
 El PIN se ingresa manualmente con el encoder, de la misma forma 
 que se indicó en la sección \ref{pinsection}.
 Si se ingresó de manera correcta, aparecerá el mensaje "valid PIN", 
 y se llegará al menú de administrador (ver sección \ref{adminmenu}).
Si se ingresó de manera incorrecta, aparecerá el mensaje "invalid PIN".
Tras 3 intentos de ingreso de PIN incorrectos, se bloquea el usuario, 
y deberá esperar un tiempo para poder reingresarlo.


 \section{Menú de Administrador}
 \label{adminmenu}
 


